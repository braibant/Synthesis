\documentclass[preprint]{sigplanconf}

\usepackage{macros}
\usepackage{lstcoq}
\usepackage{mathpartir}
\usepackage{amsfonts}
\usepackage{amsmath}

\authorinfo{Thomas Braibant}
           {Inria}
           {thomas.braibant@inria.fr}
\authorinfo{Adam Chlipala}
           {MIT}
           {adamc@csail.mit.edu}
\title{Formal verification of hardware synthesis}
\newcommand{\project}{Fe-Si}
\begin{document}
\maketitle

\begin{abstract}
  We report on the implementation of a certified compiler for an
  high-level hardware description language (HDL) called \emph{Fe-Si}
  (Featherweight Synthesis).

  Fe-Si is a simplified version of Bluespec, an HDL based on a notion
  of \emph{guarded atomic actions}. Fe-Si is defined as a
  dependently-typed deep-embedding in Coq. The target language of the
  compiler corresponds to a synthesisable subset of Verilog or VHDL.
  
  One key of our approach is that input programs to the compiler can
  be defined and proved correct inside Coq. Then, we use extraction
  and a Verilog back-end (written in OCaml) to get a certified version
  of some hardware designs.
\end{abstract}

\section*{Introduction}
Verification of hardware designs has been thoroughly investigated, and
yet, obtaining provably correct hardware of significant complexity is
usually considered challenging and time-consuming. 
%
On the one hand, a common practice in hardware verification is to take
a given design written in an hardware description language like
Verilog or VHDL, and argue about this design in a formal way using a
model checker or an SMT solver.
%
On the second hand, a completely different approach is to design
hardware via a shallow-embedding of circuits in a theorem
prover~\cite{hanna-veritas,UCAM-CL-TR-77,hunt89,vamp,certifying-circuits-in-type-theory}.
%
Yet, both kind of approach suffer from the fact that most hardware
designs are expressed in low-level register transfer languages (RTL)
like Verilog or VHDL, and that the level of abstraction they provide
may be too low to do short and meaningful proof of high-level
properties.

\medskip

To raise this level of abstraction, industry moved to \emph{hardware
  synthesis} using higher-level languages, e.g., System-C,
Esterel~\cite{DBLP:conf/birthday/Berry00} or
Bluespec~\cite{bluespec}, in which a high-level source program is
compiled to an RTL description. 
%
High-level synthesis has two benefits. 
%
First, it reduces the effort necessary to produce an hardware design.
%
Second, writing or reasoning about a high-level program is simpler
than reasoning about the (much more complicated) RTL description
generated by a compiler.
%
However, the downside of high-level synthesis is that there is no
formal guarantee that the generated circuit description behaves
exactly as prescribed by the semantics of the source
program, making verification on the high-level program useless in the
presence of compiler-introduced bugs.
%

\medskip In this paper, we investigate the formal verification of a
lightly optimizing compiler from a Bluespec-inspired language called
\project{} to RTL, quite literally applying the ideas behind the
CompCert project~\cite{Leroy-Compcert-CACM} to hardware synthesis.

\medskip

\project{} can be seen as a stripped-down and simplified version
of Bluespec: in both languages, hardware designs are described in
terms of \emph{guarded atomic actions} on storage elements. 
%
In our development, we define a (dependently-typed) deep-embedding of
the \project{} programming language in Coq using \emph{parametric
  higher-order abstract syntax (PHOAS)}~\cite{phoas-chlipala}, and
give it a semantics using an interpreter: the semantics of a program
is a Coq function that takes as inputs the current state of the
storage elements and a list of updates to be commited to this storage
elements, and produces another list of updates to be commited.
%
The one odditiy here is that this language and its semantics have a
flavour of \emph{transactional memory}, where updates to state
elements are not visible before the end of the transaction (a
time-step).
%
Our target language can be sensibly interpreted as \emph{clocked
  sequential machines}: we generate an RTL description syntactically
described as combinational definitions and next-state assignements.

\medskip

What is new about \project{} is that we embed a hardware decription
language as a domain-specific-language in Coq: circuits correspond to
a given datastructure implemented in Coq.
%
In particular, this makes it possible to use Coq as a metaprogramming
tool to describe circuits: as an example, we shall see how we can
build succint, and provably correct, description of recursive
circuits.

\paragraph{A case-study in seamless verification.}


\paragraph{Contributions.}
We summarize the contributions of this work as follows:
\begin{itemize}
\item we define a domain specific language for (minimal, high-level)
  hardware description language inside the Coq proof assistant;
\item we prove the total correctness of a compiler for this DSL that
  produces RTL code.  
\end{itemize}


%% (Note that we do not investigate yet the correctness of an
%% \emph{actual} implementation of this RTL description using the
%% synthesisable subsets of Verilog or VHDL.)

\section{From Fe-Si to RTL}
In this section, we present our source, intermediate, and target
languages, along with their static and dynamic semantics.
%
In this section, we describe these languages using ``pen and paper''
style mathematical notations, in order to avoid the clutter of Coq
formal syntax.

\section{Lightweight optimizations}
\subsection{Common sub-expression elimination}
\subsection{Using BDDs to reduce boolean expressions}

\section{Detailed example: running IMP programs on certified
  hardware.}

\section{Further discussion}

\subsection{Meta-programming features at work}

\subsection{Testing designs}
In order to gain confidenced in the fact that there is no lapse in
either the semantics of the source language, nor the semantics of the
target language, we test the simulated execution of compiled designs
against our semantics. 

Indeed, our use of program extraction is more delicate than what is
done in, e.g., CompCert, because our program transformations use
dependent types intensively. In the end, either Coq extraction
mechanism or our Verilog back-end could introduce bugs in the
generated code.

Moreover, since the long-term idea is to generate high-confidence
hardware, one should not blindly trust formally certified code that
has not been tested.

\section{Implementation}
The compiler implementation and documentation are available on- line
at: 
%
\begin{center}
  \texttt{htt://google.fr}
\end{center}
%
The size of the development can be estimated from the line counts in
Figure~\ref{fig:loc}.
\begin{figure}
  \centering  
  \caption{Size of the development (non-blank lines of code)}
  \label{fig:loc}
\end{figure}


% \category{CR-number}{subcategory}{third-level}

% \terms
% term1, term2

% \keywords
% keyword1, keyword2


\acks Acknowledgement

\bibliographystyle{abbrvnat}
\bibliography{synthesis}

\end{document}
