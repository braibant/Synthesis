\documentclass[preprint]{sigplanconf}

\usepackage{macros}
\usepackage{lstcoq}
\usepackage{mathpartir}
\usepackage{amsfonts}
\usepackage{amsmath}
\usepackage{stmaryrd}
\authorinfo{Thomas Braibant}
           {Inria}
           {thomas.braibant@inria.fr}
\authorinfo{Adam Chlipala}
           {MIT}
           {adamc@csail.mit.edu}
\title{Formal verification of hardware synthesis}
\newcommand{\project}{Fe-Si}
\newcommand{\action}{action}
\newcommand{\denote}[1]{\llbracket #1 \rrbracket}
\begin{document}
\maketitle

\begin{abstract}
  We report on the implementation of a certified compiler for an
  high-level hardware description language (HDL) called \emph{Fe-Si}
  (Featherweight Synthesis).

  Fe-Si is a simplified version of Bluespec, an HDL based on a notion
  of \emph{guarded atomic actions}. Fe-Si is defined as a
  dependently-typed deep-embedding in Coq. The target language of the
  compiler corresponds to a synthesisable subset of Verilog or VHDL.
  
  One key of our approach is that input programs to the compiler can
  be defined and proved correct inside Coq. Then, we use extraction
  and a Verilog back-end (written in OCaml) to get a certified version
  of some hardware designs.
\end{abstract}

\section*{Introduction}
Verification of hardware designs has been thoroughly investigated, and
yet, obtaining provably correct hardware of significant complexity is
usually considered challenging and time-consuming. 
%
On the one hand, a common practice in hardware verification is to take
a given design written in an hardware description language like
Verilog or VHDL, and argue about this design in a formal way using a
model checker or an SMT solver.
%
On the second hand, a completely different approach is to design
hardware via a shallow-embedding of circuits in a theorem
prover~\cite{hanna-veritas,UCAM-CL-TR-77,hunt89,vamp,certifying-circuits-in-type-theory}.
%
Yet, both kind of approach suffer from the fact that most hardware
designs are expressed in low-level register transfer languages (RTL)
like Verilog or VHDL, and that the level of abstraction they provide
may be too low to do short and meaningful proof of high-level
properties.

\medskip

To raise this level of abstraction, industry moved to \emph{hardware
  synthesis} using higher-level languages, e.g., System-C,
Esterel~\cite{DBLP:conf/birthday/Berry00} or
Bluespec~\cite{bluespec}, in which a high-level source program is
compiled to an RTL description. 
%
High-level synthesis has two benefits. 
%
First, it reduces the effort necessary to produce an hardware design.
%
Second, writing or reasoning about a high-level program is simpler
than reasoning about the (much more complicated) RTL description
generated by a compiler.
%
However, the downside of high-level synthesis is that there is no
formal guarantee that the generated circuit description behaves
exactly as prescribed by the semantics of the source
program, making verification on the high-level program useless in the
presence of compiler-introduced bugs.
%

\medskip In this paper, we investigate the formal verification of a
lightly optimizing compiler from a Bluespec-inspired language called
\project{} to RTL, quite literally applying the ideas behind the
CompCert project~\cite{Leroy-Compcert-CACM} to hardware synthesis.

\medskip

\project{} can be seen as a stripped-down and simplified version
of Bluespec: in both languages, hardware designs are described in
terms of \emph{guarded atomic actions} on storage elements. 
%
In our development, we define a (dependently-typed) deep-embedding of
the \project{} programming language in Coq using \emph{parametric
  higher-order abstract syntax (PHOAS)}~\cite{phoas-chlipala}, and
give it a semantics using an interpreter: the semantics of a program
is a Coq function that takes as inputs the current state of the
storage elements and a list of updates to be commited to this storage
elements, and produces another list of updates to be commited.
%
The one odditiy here is that this language and its semantics have a
flavour of \emph{transactional memory}, where updates to state
elements are not visible before the end of the transaction (a
time-step).
%
Our target language can be sensibly interpreted as \emph{clocked
  sequential machines}: we generate an RTL description syntactically
described as combinational definitions and next-state assignements.

\medskip

What is new about \project{} is that we embed a hardware decription
language as a domain-specific-language in Coq: circuits correspond to
a given datastructure implemented in Coq.
%
In particular, this makes it possible to use Coq as a metaprogramming
tool to describe circuits: as an example, we shall see how we can
build succint, and provably correct, description of recursive
circuits.



\paragraph{Contributions.}
We summarize the contributions of this work as follows:
\begin{itemize}
\item we demonstrate a novel use of PHOAS as a way to embed domain
  specific languages (DSLs) in the Coq proof assistant;
\item we define such a DSL for a (minimal, high-level) hardware
  description language;
\item we prove the correctness of a compiler for this DSL that
  produces RTL code. 
\end{itemize}


%% (Note that we do not investigate yet the correctness of an
%% \emph{actual} implementation of this RTL description using the
%% synthesisable subsets of Verilog or VHDL.)

\section{Overview of Fe-Si}
Fe-Si is a purely functional language built around a \emph{monad} that
makes it possible to define circuits. We start with a customary
example: we show the implementation of a half adder.
\begin{coq}
Definition hadd (a b: Var Bool) : action [] (Bool $\otimes$ Bool) :=
$\quad$do carry <- ret (andb a b; 
$\quad$do sum     <- ret (xorb a b);
$\quad$ret (carry, sum).  
\end{coq}
The typing of this circuit show that it has two input wires, and
return a tuple of values. 
%
Here, we use Coq notations to implement some syntactic sugar: we
borrow the \texttt{do}-notation to denote the monadic bind, and use
\coqe{ret} as a short-hand for return. 
% 
Our explicit use of return may seem odd. It is due to the fact that
Fe-Si has two classes of syntactic values, expressions and actions,
and that return takes as argument an
expression\footnote{Unfortunately, it is not possible to define return
  as an implicit coercion from expressions to actions.}. 

Up to this point, Fe-Si can be seen as an extension of the Lava
language, implemented in Coq rather than Haskell. Yet, using Coq as a
metalanguage offers the possibility to use dependent types in our
circuit descriptions. For instance, one can define an adder circuit of
the following type:
\begin{coq}
Definition adder n (a b: Var (Int n)): action [] (Int n) := ...
\end{coq}
In this definition, \coqe{n} of type \coqe{nat} is a formal parameter
that denotes the size of the operands and the size of the result as
well. (Here, we neglect the overflow that may occur when computing the
result.)

\paragraph{Stateful programs.}
Fe-Si also features a small set of primitives for interacting with
\emph{memory elements} that hold mutable state. In the following
snippet, we build a counter that increments its value when its input
is true.
\begin{coq}
Definition $\Phi$ := [Reg (Int n)]
Definition count n (tick: Var Bool) : action $\Phi$ (Int n) :=
$\quad$do x <- !member_0;
$\quad$do _ <- if tick then {member_0 ::= x + 1} else {ret tt}; 
$\quad$ret x. 
\end{coq}
Here, $\Phi$ is an environment that defines the set of memory elements
(in a broad sense) of the circuit. In the first line, we read the
content of register at position \coqe{member_0} in $\Phi$, and bind this
value to \coqe{x}. Then, we test the value of the input \coqe{tick},
and when it is true, we increment the value of the register. In any
case, the output is the old value of the counter.

Finally, note that the above ``if-then-else'' construct is defined
using two primitives for guarded atomic actions that are reminiscent
of transactionnal memory monads: \coqe{assert} and \coqe{orElse}. The
former aborts the current action if its argument is false. 
%
The latter takes two arguments $a$ and $b$, and first executes $a$; if
it aborts, then the effects of $a$ are discarded and $b$ is run. If
$b$ aborts too, the whole action \coqe{$a$ orElse $b$} aborts.

\paragraph{Synchronous semantics.} Recall that Fe-Si programs are
intended to describe hardware circuits. Hence, we must stress that
they are interpreted in a synchronous setting.
%
From a logical point of view the execution of a program (an atomic
action) is clocked, and at each tick of its clock, the computation of
its effects (i.e., updates to memory elements) is instantaneous; 
yet these effects are applied all at once between ticks. 
%
In particular this means that it is not possible to observe, e.g.,
partial updates to the memory elements, nor transient values in
memory.

\paragraph{From programs to circuits.} At this point, the reader may
wonder how it is possible to generate circuits in a palatable format
out of Fe-Si programs. Indeed, using Coq as a meta-language to embed
Fe-Si yields two kind of issues. First, Coq lacks any kind of I/O; and
second, a Fe-Si program may have been built using arbitrary Coq code,
including, e.g., higher-order functions or fixpoints.

Therefore, we take the following steps.  Starting from a closed Fe-Si
program \coqe{foo}, we put the following definition in given Coq file:
\begin{coq}
Definition bar := fesic foo.  
\end{coq}
Coq's extraction mechanism makes it possible to generate OCaml code
from Coq programs. Starting with the definition \coqe{bar}, it is
possible to build an OCaml file that contains all the dependencies of
\coqe{bar}, including the code of the \coqe{fesic} compiler!
%
We can then compile this OCaml file and link it to an (unverified)
back-end that pretty-prints some Verilog code that corresponds to
\coqe{bar}.
%
(We reckon that this is some devious use of the extraction mechanism,
that palliates the fact that there is currently no I/O mechanism in
Coq.)

\section{From Fe-Si to RTL}
In this section, we present our source, intermediate, and target
languages, along with their semantics.
%
% We take the liberty to describe these languages using ``pen and
% paper'' style mathematical notations, in order to avoid the clutter of
% Coq formal syntax.

\subsection{The memory model}
Fe-Si programs are meant to describe sequential circuits, whose
``memory footprint'' must be known statically. We take a declarative
approach: each state-holding element that is used in a program must be
declared. 
%
We currently have three types of memory elements: inputs, registers,
and register files. A register hold one value of a given type, while a
register file of size $n$ stores $2^n$ values of a given type. 
%
An input is a memory element that can only be read by the circuit,
and whose value is driven by the external world.
%
We show the inductive definitions of types and memory elements in
Fig.~\ref{fig:type}. 
%
We have four constructors for the type \coqe{ty} of types: \coqe{Unit}
(the unit type), \coqe{B} (Booleans), \coqe{Int} (integers of a given
size), and \coqe{Tuple} (tuples of types). The inductive definition of
memory elements (\coqe{mem}) should be self-explaining. 

\begin{figure}
  \centering
\begin{twolistings}
\begin{coq}
Inductive ty : Type :=
| Unit : ty 
| B : ty 
| Int : nat -> ty
| Tuple : list ty -> ty.     
\end{coq}&
\begin{coq}
Inductive mem : Type :=
| Input: ty ->  mem
| Reg : ty -> mem
| Regfile : nat -> ty -> mem. 
$ $
\end{coq}
\end{twolistings}
\caption{Type and memory elements}
  \label{fig:type}
\end{figure}

\newcommand{\denotety}[1]{\denote{\mathtt{#1}}_{\mathtt{ty}}}
\newcommand{\denotemem}[1]{\denote{\mathtt{#1}}_{\mathtt{mem}}}



\subsection{Fe-Si}
The definition of Fe-Si programs (\coqe{action} in the following)
takes the PHOAS approach. 
%
That is, we define an inductive type family parametrized by an
arbitrary type \coqe{V} of variables, where binders bind variables
instead of arbitrary terms (as it would be the case using HOAS), and
those variables are used explicitly via a dedicated term constructor.
%
The definition of Fe-Si syntax is split in two syntactic classes:
expressions and actions. 
%
Expressions are side-effects free, and are built from variables,
constants, and operations.
%
Actions are made of control-flow structures (assertions and
alternatives), binders, and memory operations. 

In this work, we follow an intrinsic
approach~\cite{DBLP:journals/jar/BentonHKM12}: we mix the definition
of the abstract syntax and the typing-rules from the start. That is,
the type system of the meta-language (Coq) enforces that all Fe-Si
programs are well-typed by construction.
%
Besides the obvious type-oblivious definitions (e.g., it is not
possible to add a Boolean and an integer), this means that the
definition of operations on state-holding elements requires some care.
%
Here, we use dependently-typed de Bruijn indices. 
\begin{coq}
Inductive member : list mem -> mem ->  Type :=
| member_0 : forall E t, member (t::E) t
| member_S : forall E t x, member E t -> member (x::E) t.
\end{coq}
Using the above definition, a term of type \coqe{member $\Phi$ M} denotes
the fact that the memory element \coqe{M} appears at a given position
in the environment of memory elements $\Phi$. 
%
We are now ready to present the (elided) Coq definitions of the
inductives for \coqe{expr} and \coqe{action} in Fig.~\ref{fig:fesi}.
%
(For the sake of brevity, we omit the constructors for accesses to
register files, in the syntax and, later, in the semantics. We refer
the reader to the supplementary materials for more details.)
%
Our final definition \coqe{Action} of actions is a polymorphic
function from a choice of variables to an action.

\begin{figure}
  \centering
\begin{coq}
Section t. 
Variable V: ty -> Type. Variable $\Phi$: list mem. 
Inductive expr: ty -> Type :=
| Evar : forall t (v : V t), expr t
(* operations on Booleans *)
| Eandb : expr B -> expr B -> expr B | ... 
(* operations on words *)
| Eadd : forall n, expr (Int n) -> expr (Int n) -> expr (Int n) | ... 
(* operations on tuples *)
| Efst : forall l t, expr (Tuple (t::l)) -> expr t | ...

Inductive action: ty -> Type:=
| Return: forall t, expr t -> action t
| Bind: forall t u,  action  t -> (V t -> action u) -> action u
(* control-flow *)
| OrElse: forall t, action t -> action t -> action t.
| Assert: expr B -> action Unit    
(* memory operations on registers *)
| RegRead : forall t, member $\Phi$ (Reg t) -> action t
| RegWrite: forall t, member $\Phi$ (Reg t) -> expr t -> action Unit
(* memory operations on register files, and inputs *)
| ... 
End t. 
Definition Action $\Phi$ t := forall V, action V $\Phi$ t.  
\end{coq}
  \caption{The syntax of expressions and actions}
  \label{fig:fesi}
\end{figure}

\paragraph{Semantics.}
We endow Fe-Si programs with a simple synchronous semantics:  starting
from an initial state, the execution of a Fe-Si programs corresponds
to a sequence of atomic updates to the memory elements. 
%
Each step goes as follows: reading the state, computing an update to
the state, commiting this update.
%

\begin{figure*}
  \centering
  \begin{mathpar}
    \inferrule{\Gamma \vdash e \leadsto v} {\Gamma, \Delta \vdash
      \mathtt{Return}~e \to \mathtt{Some} (v,\Delta)}
    \\
    % bind
    \inferrule{ \Gamma, \Delta_1 \vdash a \to \mathtt{None} } {\Gamma,
      \Delta_1 \vdash \mathtt{Bind}~a~f \to \mathtt{None}} \and
    \inferrule{ \Gamma, \Delta_1 \vdash a \to \mathtt{Some}~(v,
      \Delta_2) \and \Gamma, \Delta_2 \vdash f~v \to r} {\Gamma,
      \Delta_1 \vdash \mathtt{Bind}~a~f \to r}
    \\
    % assert
    \inferrule{\Gamma \vdash e \leadsto \mathtt{true}} {\Gamma, \Delta
      \vdash \mathtt{Assert}~e \to \mathtt{Some} (\mathtt{tt},\Delta)}
    \and \inferrule{\Gamma \vdash e \leadsto \mathtt{false}} {\Gamma,
      \Delta \vdash \mathtt{Assert}~e \to \mathtt{None}}
    \\
    % register
    \inferrule{\Gamma(r) = v} {\Gamma, \Delta \vdash
      \mathtt{RegRead}~r \to \mathtt{Some} (r,\Delta)} \and
    \inferrule{\Gamma \vdash e \leadsto v} {\Gamma, \Delta \vdash
      \mathtt{RegWrite}~r~e \to \mathtt{Some}
      (\mathtt{tt},\Delta\oplus(r,v))}
        
  \end{mathpar}
\caption{Dynamic semantics of Fe-Si programs}\label{fig:fesi-sem}
\end{figure*}

The reduction rules of Fe-Si programs are defined in
Fig.~\ref{fig:fesi-sem}. The judgement $\Gamma, \Delta \vdash a \to r$
reads ``in the state $\Gamma$ and with the partial update $\Delta$,
evaluating $a$ produces the result $r$'', where $r$ is either
\coqe{None} (meaning that the action aborted), or %
\coqe{Some (v, $\Delta$)} (meaning that the action returned the value
\coqe{v} and the partial update $\Delta$). 
%
The rules for the reduction of \coqe{Bind} may seem peculiar, in that
they do not modify the state $\Gamma$. 
%
This is because the PHOAS approach makes it possible to manipulate
closed-terms: we hijack Coq's $\beta$-reduction when dealing with
$f~v$, and we do not need to have an explicit treatment of the binders
nor the reduction environments.

There are two peculiarities here: first, following the definition of
$\oplus$, if two values are written to a memory element, only the
first one (in program order) is commited; second, reading a register
yield the value that was held at the beginning of the time step. 
%
(While it may seem an odd choice for a general purpose programming
language, it is meaningful when it comes to hardware. We come back to
this point in \S\ref{sec:discussion}.)

Finally, we define a wrapper function that computes the next state of
the memory elements, using the aforementioned evaluation relation
(starting with an empty partial update). 
\begin{coq}
Definition Next {t} $\Phi$ (st: $\denote{\Phi}$) (A : Action Phi t) : $\denote{\Phi}$ := ...
\end{coq}

\subsection{RTL} 
We now turn to the presentation of our target language, which sits at
the register-transfer level. At this level, synchronous circuit can be
faithfully described as a set of state holding elements, and a
next-state function, implemented using combinational
logic~\cite{DBLP:journals/cj/Gordon02}.
%
Therefore, The definition of RTL programs (\coqe{block} in the
following) is quite simple: a program is simply a telescope of
expressions (combinational operations, or reads from state holding
elements), with a list of effects (i.e, writes to state holding
elements) at the end. 

We show the definition of expressions, telescopes, and blocks in
Fig.~\ref{fig:rtl}. 
%
The definition of expressions is similar to the one we used for Fe-Si,
excepts that we have constructors for reads from memory elements, and
that we moved to ``three-adress code''.
%
(That is, operands are variables, rather than arbitrary expressions.)
%
A telescope (type \coqe{scope A}) can be thought of as a sequence of
bindings of expressions, with an element of type \coqe{A} at the
end (\coqe{A} is instantiated later with a list of effects.)
%
Intuitively, the first binding of a telescope can only read from
memory elements; the second binding may use the first value, or read
from memory elements; and so on and so forth.

A \coqe{block} is a telescope, with three elements at the end: a
guard, a return-value, and a (dependently-typed) list of effects. 
%
The value of the guard (a Boolean) is equal to true when the
return-value and the effects are valid (i.e., the effects should be
commited to memory); and false otherwise.
%
The return-value denotes the outputs of the circuits. 
%
The datatype \coqe{effects} encode, for each memory element of the
list $\Phi$, either an effect (a write of the right type), or None
(meaning that this memory element is never written to). (For the sake
of brevity, we omit the particular definition of dependently-typed
heterogeneous lists \coqe{DList.T} that we use here.)


\begin{figure}
  \centering
\begin{coq}
Section t. 
Variable V: ty -> Type. Variable $\Phi$: list mem. 
Inductive expr: ty -> Type :=
| Evar : forall t (v : V t), expr t
(* read from memory elements *)
| Einput : forall t, member $\Phi$ (Input t) -> expr t
| Eread_r : forall t, member $\Phi$ (Reg t) -> expr t
| Eread_rf : forall n t, member $\Phi$ (Regfile n t) -> V (Int n) -> expr t
(* operations on Booleans *)
| Emux : forall t, V B -> V t -> V t -> expr t
| Eandb : V B -> V B -> V B | ... 
(* operations on words *)
| Eadd : forall n, V (Int n) -> V (Int n) -> expr (Int n) | ... 
(* operations on tuples *)
| Efst : forall l t, V (Tuple (t::l)) -> expr t | ...

Inductive scope (A : Type): Type :=
| Send : A -> scope A
| Sbind : forall (t: ty), expr t -> (V t -> scope A) -> scope A. 

Inductive write : mem -> Type :=
| WR : forall t, V t -> V Tbool -> write (Reg t)
| WRF : forall n t, V t -> V (Int n) -> V B ->  write (Regfile n t). 
     
Definition effects := DList.T (option $\circ$ write) $\Phi$. 
Definition block t := scope (V B * V t *  effects).         
End t.
Definition Block $\Phi$ t := forall V, block $\Phi$ V t.
\end{coq}
  \caption{RTL programs with three-adress code expressions}
  \label{fig:rtl}
\end{figure}

\paragraph{Semantics.} We now turn to define the semantics of our RTL
language. 
%
To define  evaluation 
We first define a denotation for closed expressions (in the
same way as we did at the source-level, except that it is not a
recursive definition).
\begin{coq}
Variable $\Gamma$: $\denote{\Phi}$. 
Definition eval_expr (t : ty) (e : expr $\denotety{.}$ t) : $\denotety{.}$:=
match e with
| Evar t v => v
| Einput t v => DList.get v $\Gamma$
| Eread  t v =>  DList.get v $\Gamma$
| Eread_rf n t v adr => Regfile.get (DList.get v $\Gamma$) adr
| Emux t b x y => if b then x else y 
| Eandb a b => andb a b | ...
| Eadd n a b => Word.add a b  | ...
| Efst l t e => Tuple.fst e | ...
end. 
\end{coq}

We now turn to define the (simple) denotation of telescopes: it is a
simple recursive function that evaluates bindings in order, and apply
an arbitrary function on the final (closed) object. 
\begin{coq}
Fixpoint eval_scope {A B} (F : A -> B) (T : scope $\denotety{.}$ A) :=
match T with 
| Send X => F X
| Sbind t e cont => eval_scope F (cont (eval_expr t e))
end.   
\end{coq}
%
The final piece that we need is the denotation that corresponds to the
\coqe{write} type. This function takes as argument a single effect,
the initial state of this memory location, and either returns a new
state for this memory location, or returns \coqe{None}, meaning that
location is left in its previous state.
\begin{coq}
Definition eval_effect (m : mem) : 
$\qquad$option (write $\denotety{.}$ m) -> $\denotemem{m}$ -> (option $\denotemem{m}$) := ... 
\end{coq}
%
Using all this pieces, we will directly jump to the final next-state
function. Doing so, we elide some of the wrapping that need to be done
in our Coq development, for the sake of clarity.
\begin{coq}
Definition Next {t} $\Phi$ ($\Gamma$: $\denote{\Phi}$) (B : Block $\Phi$ t) : $\denote{\Phi}$ := ...
\end{coq}

\subsection{Compiling Fe-Si to RTL} 
Our syntactic translation from Fe-Si to RTL is driven by the fact that
our RTL language does not allow clashing assignements: syntactically,
each register and register-file is updated at most one \coqe{write}
expression.
%
With a wrinkle, we could say that we move to a language with
\emph{single-static assignements}. 

\paragraph{From control-flow to data-flow.}To do so, we have to transform
the control-flow (the \coqe{Assert} and \coqe{OrElse}) of Fe-Si
programs into data-flow.
%
We can do that in hardware, because circuits are inherently parallel:
for instance, the circuit that compute the result of the conditional
expression \mbox{\coqe{e ? a : b}} is a circuit that computes the value of
\coqe{a} and the value of \coqe{b} in parallel, and then uses the
value of \coqe{e} to select the right value for the whole expression. 

Our first compilation pass transforms Fe-Si programs into an
intermediate language called IR that implements A-normal form. That
is, we assign names to every intermediate computation.
%
In order to do so, we also have to resolve the control-flow. To be
more specific about the latter, given an expression like
\begin{coq}
do x <- (A OrElse B); ... 
\end{coq}
we want to know statically to what value \coqe{x} needs to be bound
and when this value is \emph{valid}. 
%
In this particular case, we remark that if \coqe{A} yields a value
$v_A$ which is valid, then \coqe{x} needs to be bound to $v_A$; if
\coqe{A} yields a value that is invalid, then \coqe{x} needs to be
bound to the value returned by \coqe{B}. In any case, the value bound
in \coqe{x} is valid whenever the value returned by \coqe{A} or the
value returned by \coqe{B} is valid.

More generally, our compilation function takes as argument an
arbitrary function, and returns a telescope\footnote{Here, we use
  Coq's module system to discriminate between various incarnations of
  our telescope and our expression data-type, and to provide
  annotations for the reader.}:
\begin{coq}
Fixpoint compile t (a : action $\Phi$ V t) : 
$\qquad$IR.scope (IR.expr V B * V t * list IR.neffect) := ...
\end{coq}
The first element of the tuple that is bound by the
telescope is the \emph{guard}, that denotes the validity of the
following components of the tuple.
%
The second element of the tuple is a variable, that is bound by the
telescope to denote the value that was returned by the action. 
%
The third element of the tuple is a list of \emph{nested effects},
which are a lax version of the effects that exist at the \coqe{RTL}
level.
\begin{coq}
Inductive neffect  : Type :=
| NGuard : forall (guard : expr V B), list neffect -> neffect 
| NRegW : forall t, member $\Phi$ (Reg t) -> V t -> neffect
| NRegfileW : ... 
\end{coq}
The rationale behind the \coqe{neffect} data-type is to represent
trees of conditional blocks, with writes to state-holding elements at
the leaves. (Using this data-type, several paths in such a tree may
lead to a write to a given memory location; in this case, we use a
notion of program order to discriminate between clashing
assignements. Note also that the \coqe{list neffect} that sits at the
end of the telescope in the \coqe{compile} is a shortcut for a
degenerate \coqe{NGuard} whose guard is true.)

The compilation by itself is quite straighforward. We give the code
that corresponds to the \coqe{Assert} and the \coqe{Bind} cases below
% \begin{coq}
% | Bind t u A F =>
% $\quad$[< rA, gA, eA >] :- compile _ A; 
% $\quad$[< rB, gB, eB >] :- compile _ (F rA); 
% $\quad$Send (rB, andb gA gB, List.app eA eB) 
% \end{coq}
\begin{coq}
| Return t e => Sbind e (fun x => Send (true, Evar x, nil))
| Bind t u A F =>
$\quad$ compile _ A $\rhd$ (fun (rA, gA, eA) =>
$\quad$ compile _ (F rA) $\rhd$ (fun (rB, gB, eB) =>
$\quad$ Send (rB, andb gA gB, List.app eA eB))) 
| Assert e => Sbind e (fun x => Send ( Evar x, tt, nil))
\end{coq}
where $\rhd$ corresponds to the composition of telescopes, and
\coqe{andb} is a ``smart constructor'' of the expresssion data-types,
that avoids building degenerate cases.

\paragraph{Linearizing the effects} Our second compilation pass
flattens the nested effects that were introduced in IR, and targets
our second intermediate language, IR2. (The IR2 language is similar to
the RTL language in all respects, except that it is not yet in
three-adress code.)

The idea of this translation is to associate two values to each
register: a \emph{data} value (the value that ought to be written) and
a \emph{write-enable} value. The data value is commited (i.e., stored)
to the register if the write-enable Boolean is true.
%
Similarly, we associate three values to each register-file: a data, an
address, and a write-enable. The data is stored to the field of the
register file selected by the address if the write-enable is true.  

The heart of this translation is a \coqe{merge} function that takes
two \coqe{write} of the same type, and returns a telescope that
encapsulates a single \coqe{write}: 
\begin{coq}
Definition merge s (a b : write s): scope (option (write s)) := ...   
\end{coq}
Despite requiring a bit of dependent-types hackery\footnote{The
  reader may wonder why we need an option here; the answer is that we
  could do without it, at the price of a more complicated definition
  for \coqe{merge} that uses the fact that the type
%
  \mbox{\coqe{write (Input t)}} is not inhabited.}, the definition of
\coqe{merge} is the expected one.
%
For instance, in the register case, given $(v_a,we_a)$ (resp. $(v_b,
we_b)$) the value and the write-enable that corresponds to \coqe{a},
the write-enable that corresponds to the merge of \coqe{a} and
\coqe{b} is $we_a || we_b$, and the associated data is \mbox{$we_a~?~v_a :
v_b$}.

% We come back to the counter we used in the introduction, that needs to
% be sligthly modified to get a closed Fe-Si program. 
% \begin{coq}
% Definition $\Phi$ := [Reg (Int n); Input Bool]
% Definition count n : action $\Phi$ (Int n) :=
% $\quad$do x <- !member_0; do tick <- input (member_S member_0);
% $\quad$do _ <- if tick then {member_0 ::= x + 1} else {ret tt}; 
% $\quad$ret x. 
% \end{coq}

\paragraph{Moving to RTL.} The only thing that remains to be done is
the translation from IR2 to RTL, which amounts to a simple
transformation into three-address code. 

\section{Lightweight optimizations}
In this section, we describe two optimizations on the RTL
language. The first one is syntactic version of common sub-expression
elimination, intended to reduce the amount of bindings, and introduce
more sharing. The second is a semantic common sub-expression
elimination that aims to reduce the size of the Boolean formula that
were generated in the previous translation passes. 

\subsection{CSE}
We perform syntactic common sub-expression elimination using a simple
recursive traversal of RTL programs. Indeed, since there is no
iteration involved, there is no need to perform any kind of data-flow
analysis. 

We follow the approach used by
Chlipala~\cite{DBLP:conf/popl/Chlipala10} to perform CSE in previous
work. That is, we tag each variable that is used in the program with a
symbolic value, as defined below.
\begin{coq}
Inductive sval: ty -> Type := ...  
\end{coq}

Contrary to our previous transformations that were just ``pushing
variables around'' for each possible choice of variable representation
\coqe{V}, here we need to tag variables with their symbolic values. 
% 
%
Then, CSE goes as follows. We fold through a telescope and maintain a
mapping from symbolic values to variables. For each binder of the
telescope, we compute the symbolic representation of the expression
that is bound. 
%
If this symbolic value is already in the map, we avoid the creation of
an extraneous binder. Otherwise, we do create a new binder, and extend
our association list accordingly. 
 
% Between closure
% conversion and flattening, we perform intrapro- cedural common
% subexpression elimination (CSE) on closed pro- grams. Since our
% languages have no intraprocedural iteration con- structs, there is no
% need to perform dataflow analysis. Instead, a single recursive
% traversal of a program suffices. The optimization still simplifies
% cases like application of a known function, where it is possible to
% avoid building a closure.

% As we descend into a program’s structure, we maintain a map- ping from
% variables to symbolic values, as defined below.  Symbolic values s ::=
% #n | c | () | s, s | inl(s) | inr(s) Values not built from the basic
% constant, unit, product, and sum constructors are represented with
% symbolic variables #n, where a fresh n is generated for each new
% input-program variable that cannot be determined to have more specific
% structure.  The purpose of CSE is to remove some redundant bindings
% and case analyses. This transformation may sound complicated enough to
% require conversion of input programs to first-order form to analyze
% them. However, it is possible to implement CSE in an elegant
% higher-order way. In translating a parametric program P , we must
% produce a CSE’d version of it for each possible variable
% representation var. Our solution is to do so by instantiating P at
% variable type var * sval, where sval is the type of symbolic values s.

% Thus, each variable is tagged with a symbolic representation, and this
% representation may be accessed directly at use sites. The main
% translation maintains a mapping from symbolic values to vari-
% ables. We use this mapping to simplify case expressions with dis-
% criminees that we see statically are either inl or inr. When proceed-
% ing under a let binder, the translation evaluates the bound expres-
% sion symbolically. If the result is in the map, we avoid creating a
% new binder in the translation. Instead, we apply the binder body,
% which is a function over variable/value pairs, to the variable that
% our map associates with the appropriate symbolic value, paired with
% that value. If the value we are binding is not found in the map, we do
% create a new binder, and, in the recursive call inside the binder’s
% scope, we add the new variable to the symbolic map.  The main
% correctness theorem for this translation is proved very similarly to
% the main theorem for CPS conversion. The proof can be a bit simpler
% because we need no value compatibility relation; CSE has no effect on
% the values that appear during program evaluation.  We prove the main
% theorem with about 20 lines of tactic code for performing appropriate
% case analyses, applying IHes and a lemma about primops, and
% materializing known facts about variables men- tioned in expression
% equivalence derivations.

\subsection{Using BDDs to reduce boolean expressions}

\section{Detailed example: running IMP programs on certified
  hardware.}

\section{Further discussion}\label{sec:discussion}
\begin{itemize}
\item Limitation: each register file allows for at most one write
  action in each atomic update step; yet, any number of reads are
  allowed. We could lift this limitation to allow multiple write
  to be performed at the same time, or to limit the number of reads. 

\end{itemize}
\subsection{Meta-programming features at work}

\subsection{Testing designs}
In order to gain confidenced in the fact that there is no lapse in
either the semantics of the source language, nor the semantics of the
target language, we test the simulated execution of compiled designs
against our semantics. 

Indeed, our use of program extraction is more delicate than what is
done in, e.g., CompCert, because our program transformations use
dependent types intensively. In the end, either Coq extraction
mechanism or our Verilog back-end could introduce bugs in the
generated code.

Moreover, since the long-term idea is to generate high-confidence
hardware, one should not blindly trust formally certified code that
has not been tested.

\section{Implementation}
The compiler implementation and documentation are available on- line
at: 
%
\begin{center}
  \texttt{htt://google.fr}
\end{center}
%
The size of the development can be estimated from the line counts in
Figure~\ref{fig:loc}.
\begin{figure}
  \centering  
  \caption{Size of the development (non-blank lines of code)}
  \label{fig:loc}
\end{figure}


% \category{CR-number}{subcategory}{third-level}

% \terms
% term1, term2

% \keywords
% keyword1, keyword2


\acks Acknowledgement

\bibliographystyle{abbrvnat}
\bibliography{synthesis}

\end{document}
